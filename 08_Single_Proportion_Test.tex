% Options for packages loaded elsewhere
\PassOptionsToPackage{unicode}{hyperref}
\PassOptionsToPackage{hyphens}{url}
%
\documentclass[
]{article}
\usepackage{amsmath,amssymb}
\usepackage{iftex}
\ifPDFTeX
  \usepackage[T1]{fontenc}
  \usepackage[utf8]{inputenc}
  \usepackage{textcomp} % provide euro and other symbols
\else % if luatex or xetex
  \usepackage{unicode-math} % this also loads fontspec
  \defaultfontfeatures{Scale=MatchLowercase}
  \defaultfontfeatures[\rmfamily]{Ligatures=TeX,Scale=1}
\fi
\usepackage{lmodern}
\ifPDFTeX\else
  % xetex/luatex font selection
\fi
% Use upquote if available, for straight quotes in verbatim environments
\IfFileExists{upquote.sty}{\usepackage{upquote}}{}
\IfFileExists{microtype.sty}{% use microtype if available
  \usepackage[]{microtype}
  \UseMicrotypeSet[protrusion]{basicmath} % disable protrusion for tt fonts
}{}
\makeatletter
\@ifundefined{KOMAClassName}{% if non-KOMA class
  \IfFileExists{parskip.sty}{%
    \usepackage{parskip}
  }{% else
    \setlength{\parindent}{0pt}
    \setlength{\parskip}{6pt plus 2pt minus 1pt}}
}{% if KOMA class
  \KOMAoptions{parskip=half}}
\makeatother
\usepackage{xcolor}
\usepackage[margin=1in]{geometry}
\usepackage{color}
\usepackage{fancyvrb}
\newcommand{\VerbBar}{|}
\newcommand{\VERB}{\Verb[commandchars=\\\{\}]}
\DefineVerbatimEnvironment{Highlighting}{Verbatim}{commandchars=\\\{\}}
% Add ',fontsize=\small' for more characters per line
\usepackage{framed}
\definecolor{shadecolor}{RGB}{248,248,248}
\newenvironment{Shaded}{\begin{snugshade}}{\end{snugshade}}
\newcommand{\AlertTok}[1]{\textcolor[rgb]{0.94,0.16,0.16}{#1}}
\newcommand{\AnnotationTok}[1]{\textcolor[rgb]{0.56,0.35,0.01}{\textbf{\textit{#1}}}}
\newcommand{\AttributeTok}[1]{\textcolor[rgb]{0.13,0.29,0.53}{#1}}
\newcommand{\BaseNTok}[1]{\textcolor[rgb]{0.00,0.00,0.81}{#1}}
\newcommand{\BuiltInTok}[1]{#1}
\newcommand{\CharTok}[1]{\textcolor[rgb]{0.31,0.60,0.02}{#1}}
\newcommand{\CommentTok}[1]{\textcolor[rgb]{0.56,0.35,0.01}{\textit{#1}}}
\newcommand{\CommentVarTok}[1]{\textcolor[rgb]{0.56,0.35,0.01}{\textbf{\textit{#1}}}}
\newcommand{\ConstantTok}[1]{\textcolor[rgb]{0.56,0.35,0.01}{#1}}
\newcommand{\ControlFlowTok}[1]{\textcolor[rgb]{0.13,0.29,0.53}{\textbf{#1}}}
\newcommand{\DataTypeTok}[1]{\textcolor[rgb]{0.13,0.29,0.53}{#1}}
\newcommand{\DecValTok}[1]{\textcolor[rgb]{0.00,0.00,0.81}{#1}}
\newcommand{\DocumentationTok}[1]{\textcolor[rgb]{0.56,0.35,0.01}{\textbf{\textit{#1}}}}
\newcommand{\ErrorTok}[1]{\textcolor[rgb]{0.64,0.00,0.00}{\textbf{#1}}}
\newcommand{\ExtensionTok}[1]{#1}
\newcommand{\FloatTok}[1]{\textcolor[rgb]{0.00,0.00,0.81}{#1}}
\newcommand{\FunctionTok}[1]{\textcolor[rgb]{0.13,0.29,0.53}{\textbf{#1}}}
\newcommand{\ImportTok}[1]{#1}
\newcommand{\InformationTok}[1]{\textcolor[rgb]{0.56,0.35,0.01}{\textbf{\textit{#1}}}}
\newcommand{\KeywordTok}[1]{\textcolor[rgb]{0.13,0.29,0.53}{\textbf{#1}}}
\newcommand{\NormalTok}[1]{#1}
\newcommand{\OperatorTok}[1]{\textcolor[rgb]{0.81,0.36,0.00}{\textbf{#1}}}
\newcommand{\OtherTok}[1]{\textcolor[rgb]{0.56,0.35,0.01}{#1}}
\newcommand{\PreprocessorTok}[1]{\textcolor[rgb]{0.56,0.35,0.01}{\textit{#1}}}
\newcommand{\RegionMarkerTok}[1]{#1}
\newcommand{\SpecialCharTok}[1]{\textcolor[rgb]{0.81,0.36,0.00}{\textbf{#1}}}
\newcommand{\SpecialStringTok}[1]{\textcolor[rgb]{0.31,0.60,0.02}{#1}}
\newcommand{\StringTok}[1]{\textcolor[rgb]{0.31,0.60,0.02}{#1}}
\newcommand{\VariableTok}[1]{\textcolor[rgb]{0.00,0.00,0.00}{#1}}
\newcommand{\VerbatimStringTok}[1]{\textcolor[rgb]{0.31,0.60,0.02}{#1}}
\newcommand{\WarningTok}[1]{\textcolor[rgb]{0.56,0.35,0.01}{\textbf{\textit{#1}}}}
\usepackage{graphicx}
\makeatletter
\def\maxwidth{\ifdim\Gin@nat@width>\linewidth\linewidth\else\Gin@nat@width\fi}
\def\maxheight{\ifdim\Gin@nat@height>\textheight\textheight\else\Gin@nat@height\fi}
\makeatother
% Scale images if necessary, so that they will not overflow the page
% margins by default, and it is still possible to overwrite the defaults
% using explicit options in \includegraphics[width, height, ...]{}
\setkeys{Gin}{width=\maxwidth,height=\maxheight,keepaspectratio}
% Set default figure placement to htbp
\makeatletter
\def\fps@figure{htbp}
\makeatother
\setlength{\emergencystretch}{3em} % prevent overfull lines
\providecommand{\tightlist}{%
  \setlength{\itemsep}{0pt}\setlength{\parskip}{0pt}}
\setcounter{secnumdepth}{-\maxdimen} % remove section numbering
\ifLuaTeX
  \usepackage{selnolig}  % disable illegal ligatures
\fi
\IfFileExists{bookmark.sty}{\usepackage{bookmark}}{\usepackage{hyperref}}
\IfFileExists{xurl.sty}{\usepackage{xurl}}{} % add URL line breaks if available
\urlstyle{same}
\hypersetup{
  pdftitle={Hypothesis Test for a Single Proportion (The Math Sorcerer)},
  pdfauthor={Luis Cárceles},
  hidelinks,
  pdfcreator={LaTeX via pandoc}}

\title{Hypothesis Test for a Single Proportion (The Math Sorcerer)}
\author{Luis Cárceles}
\date{2024-04-01}

\begin{document}
\maketitle

See the
\href{https://www.youtube.com/playlist?list=PLO1y6V1SXjjMIJ7HK2U7qFvoLpnNQfbuF}{YouTube
play list} by The Math Sorcerer.

\hypertarget{introduction}{%
\subsection{Introduction}\label{introduction}}

The statistic to be used in a single proportion test is

\[z=\frac{\hat{p}-p_{0}}{\sqrt{\frac{p_{0}(1-p_{0})}{n}}}\]

where \(\hat{p}\) is the sample proportion, \(p_{0}\) is the expected
proportion and \(n\) is the sample size.

It turns out that z is distributed N(0,1)

Hypothesis test using R: Let n be the sample size. If \(n\leq30\), we
use \textbf{\emph{binom.test}}. If \(n>30\) we will use
\textbf{\emph{prop.test}}

\hypertarget{problems}{%
\subsection{Problems}\label{problems}}

\hypertarget{problem-1}{%
\subsubsection{Problem 1}\label{problem-1}}

Suppose 222 subjects are treated with a drug that is used to treat pain
and 53 of them developed nausea. Use a 0.1 significance level to test
the claim that more than 20\% of users develop nausea.

Solution:

\begin{enumerate}
\def\labelenumi{\arabic{enumi}.}
\tightlist
\item
  State null and alternate Hypothesis:
\end{enumerate}

\begin{itemize}
\tightlist
\item
  \(H_{0}\): \(p=0.20\)
\item
  \(H_{1}\): \(p>0.20\) (Right tailed test)
\end{itemize}

\begin{enumerate}
\def\labelenumi{\arabic{enumi}.}
\setcounter{enumi}{1}
\tightlist
\item
  Compute the Test Statistic
\end{enumerate}

\begin{Shaded}
\begin{Highlighting}[]
\NormalTok{p10}\OtherTok{\textless{}{-}}\FloatTok{0.20}
\NormalTok{X1}\OtherTok{\textless{}{-}}\DecValTok{53}
\NormalTok{n1}\OtherTok{\textless{}{-}}\DecValTok{222}
\NormalTok{p1bar}\OtherTok{\textless{}{-}}\NormalTok{X1}\SpecialCharTok{/}\NormalTok{n1}
\NormalTok{z1}\OtherTok{\textless{}{-}}\NormalTok{(p1bar}\SpecialCharTok{{-}}\NormalTok{p10)}\SpecialCharTok{/}\FunctionTok{sqrt}\NormalTok{(p10}\SpecialCharTok{*}\NormalTok{(}\DecValTok{1}\SpecialCharTok{{-}}\NormalTok{p10)}\SpecialCharTok{/}\NormalTok{n1)}
\FunctionTok{sprintf}\NormalTok{(}\StringTok{"The value of the z statistic is: \%f"}\NormalTok{, z1)}
\end{Highlighting}
\end{Shaded}

\begin{verbatim}
## [1] "The value of the z statistic is: 1.442986"
\end{verbatim}

\begin{enumerate}
\def\labelenumi{\arabic{enumi}.}
\setcounter{enumi}{2}
\tightlist
\item
  Compute the p-value
\end{enumerate}

\begin{Shaded}
\begin{Highlighting}[]
\NormalTok{p1\_value}\OtherTok{\textless{}{-}}\DecValTok{1}\SpecialCharTok{{-}}\FunctionTok{pnorm}\NormalTok{(z1, }\DecValTok{0}\NormalTok{, }\DecValTok{1}\NormalTok{)}
\FunctionTok{sprintf}\NormalTok{(}\StringTok{"The p{-}value of the test is: \%f"}\NormalTok{, p1\_value)}
\end{Highlighting}
\end{Shaded}

\begin{verbatim}
## [1] "The p-value of the test is: 0.074512"
\end{verbatim}

\begin{enumerate}
\def\labelenumi{\arabic{enumi}.}
\setcounter{enumi}{3}
\tightlist
\item
  Make a Decision
\end{enumerate}

p-value is greater than the level of significance (0.1) then we FAIL to
reject \(H_{0}\).

\begin{enumerate}
\def\labelenumi{\arabic{enumi}.}
\setcounter{enumi}{4}
\tightlist
\item
  Interpretation of the results
\end{enumerate}

At the 1\% of significance level there's not sufficient evidence to
support \(H_{1}\) (our claim)

Extra. Using R for Hypothesis Test

\begin{Shaded}
\begin{Highlighting}[]
\FunctionTok{prop.test}\NormalTok{(}\AttributeTok{x=}\DecValTok{53}\NormalTok{, }\AttributeTok{n=}\DecValTok{222}\NormalTok{, }\AttributeTok{p=}\FloatTok{0.20}\NormalTok{, }\AttributeTok{alternative=}\StringTok{"greater"}\NormalTok{)}
\end{Highlighting}
\end{Shaded}

\begin{verbatim}
## 
##  1-sample proportions test with continuity correction
## 
## data:  53 out of 222, null probability 0.2
## X-squared = 1.8471, df = 1, p-value = 0.08706
## alternative hypothesis: true p is greater than 0.2
## 95 percent confidence interval:
##  0.1929251 1.0000000
## sample estimates:
##         p 
## 0.2387387
\end{verbatim}

\hypertarget{problem-2}{%
\subsubsection{Problem 2}\label{problem-2}}

In a study of cell phone usage and brain hemispheric dominance, an
internet survey was e-mailed to 6983 subjects randomly selected from an
online group involved with ears. There were 1317 surveys returned. Use a
0.01 significance level to test the claim that the return rate is less
than 20\%. Use the p-value method and use the normal distribution as an
approximation to the binomial distribution.

Solution:

\begin{enumerate}
\def\labelenumi{\arabic{enumi}.}
\tightlist
\item
  State null and alternate Hypothesis:
\end{enumerate}

\begin{itemize}
\tightlist
\item
  \(H_{0}\): \(p=0.20\)
\item
  \(H_{1}\): \(p<0.20\) (Left tailed test)
\end{itemize}

\begin{enumerate}
\def\labelenumi{\arabic{enumi}.}
\setcounter{enumi}{1}
\tightlist
\item
  Compute the Test Statistic
\end{enumerate}

\begin{Shaded}
\begin{Highlighting}[]
\NormalTok{p20}\OtherTok{\textless{}{-}}\FloatTok{0.20}
\NormalTok{X2}\OtherTok{\textless{}{-}}\DecValTok{1317}
\NormalTok{n2}\OtherTok{\textless{}{-}}\DecValTok{6983}
\NormalTok{p2bar}\OtherTok{\textless{}{-}}\NormalTok{X2}\SpecialCharTok{/}\NormalTok{n2}
\NormalTok{z2}\OtherTok{\textless{}{-}}\NormalTok{(p2bar}\SpecialCharTok{{-}}\NormalTok{p20)}\SpecialCharTok{/}\FunctionTok{sqrt}\NormalTok{(p20}\SpecialCharTok{*}\NormalTok{(}\DecValTok{1}\SpecialCharTok{{-}}\NormalTok{p20)}\SpecialCharTok{/}\NormalTok{n2)}
\FunctionTok{sprintf}\NormalTok{(}\StringTok{"The value of the z statistic is: \%f"}\NormalTok{, z2)}
\end{Highlighting}
\end{Shaded}

\begin{verbatim}
## [1] "The value of the z statistic is: -2.381398"
\end{verbatim}

\begin{enumerate}
\def\labelenumi{\arabic{enumi}.}
\setcounter{enumi}{2}
\tightlist
\item
  Compute the p-value
\end{enumerate}

\begin{Shaded}
\begin{Highlighting}[]
\NormalTok{p2\_value}\OtherTok{\textless{}{-}}\FunctionTok{pnorm}\NormalTok{(z2, }\DecValTok{0}\NormalTok{, }\DecValTok{1}\NormalTok{)}
\FunctionTok{sprintf}\NormalTok{(}\StringTok{"The p{-}value of the test is: \%f"}\NormalTok{, p2\_value)}
\end{Highlighting}
\end{Shaded}

\begin{verbatim}
## [1] "The p-value of the test is: 0.008624"
\end{verbatim}

\begin{enumerate}
\def\labelenumi{\arabic{enumi}.}
\setcounter{enumi}{3}
\tightlist
\item
  Make a Decision
\end{enumerate}

p-value is lower than the level of significance (0.01), then we reject
\(H_{0}\).

\begin{enumerate}
\def\labelenumi{\arabic{enumi}.}
\setcounter{enumi}{4}
\tightlist
\item
  Interpretation of the results
\end{enumerate}

At the 1\% of significance level there's sufficient evidence to support
\(H_{1}\) (our claim)

Extra. Using R for Hypothesis Test

\begin{Shaded}
\begin{Highlighting}[]
\FunctionTok{prop.test}\NormalTok{(}\AttributeTok{x=}\DecValTok{1317}\NormalTok{, }\AttributeTok{n=}\DecValTok{6983}\NormalTok{, }\AttributeTok{p=}\FloatTok{0.20}\NormalTok{, }\AttributeTok{alternative=}\StringTok{"less"}\NormalTok{)}
\end{Highlighting}
\end{Shaded}

\begin{verbatim}
## 
##  1-sample proportions test with continuity correction
## 
## data:  1317 out of 6983, null probability 0.2
## X-squared = 5.6, df = 1, p-value = 0.00898
## alternative hypothesis: true p is less than 0.2
## 95 percent confidence interval:
##  0.0000000 0.1964937
## sample estimates:
##         p 
## 0.1886009
\end{verbatim}

\hypertarget{problem-3}{%
\subsubsection{Problem 3}\label{problem-3}}

Consider a drug testing company that provides a test for marijuana
usage. Among 321 tested subjects, results from 29 subjects were wrong
(either a false positive or a false negative). Use a 0.10 significance
level to test the claim that less than 10 percent of the test results
are wrong.

Solution:

\begin{enumerate}
\def\labelenumi{\arabic{enumi}.}
\tightlist
\item
  State null and alternate Hypothesis:
\end{enumerate}

\begin{itemize}
\tightlist
\item
  \(H_{0}\): \(p=0.10\)
\item
  \(H_{1}\): \(p<0.10\) (Left tailed test)
\end{itemize}

\begin{enumerate}
\def\labelenumi{\arabic{enumi}.}
\setcounter{enumi}{1}
\tightlist
\item
  Compute the Test Statistic
\end{enumerate}

\begin{Shaded}
\begin{Highlighting}[]
\NormalTok{p30}\OtherTok{\textless{}{-}}\FloatTok{0.10}
\NormalTok{X3}\OtherTok{\textless{}{-}}\DecValTok{29}
\NormalTok{n3}\OtherTok{\textless{}{-}}\DecValTok{321}
\NormalTok{p3bar}\OtherTok{\textless{}{-}}\NormalTok{X3}\SpecialCharTok{/}\NormalTok{n3}
\NormalTok{z3}\OtherTok{\textless{}{-}}\NormalTok{(p3bar}\SpecialCharTok{{-}}\NormalTok{p30)}\SpecialCharTok{/}\FunctionTok{sqrt}\NormalTok{(p30}\SpecialCharTok{*}\NormalTok{(}\DecValTok{1}\SpecialCharTok{{-}}\NormalTok{p30)}\SpecialCharTok{/}\NormalTok{n3)}
\FunctionTok{sprintf}\NormalTok{(}\StringTok{"The value of the z statistic is: \%f"}\NormalTok{, z3)}
\end{Highlighting}
\end{Shaded}

\begin{verbatim}
## [1] "The value of the z statistic is: -0.576750"
\end{verbatim}

\begin{enumerate}
\def\labelenumi{\arabic{enumi}.}
\setcounter{enumi}{2}
\tightlist
\item
  Compute the p-value
\end{enumerate}

\begin{Shaded}
\begin{Highlighting}[]
\NormalTok{p3\_value}\OtherTok{\textless{}{-}}\FunctionTok{pnorm}\NormalTok{(z3, }\DecValTok{0}\NormalTok{, }\DecValTok{1}\NormalTok{)}
\FunctionTok{sprintf}\NormalTok{(}\StringTok{"The p{-}value of the test is: \%f"}\NormalTok{, p3\_value)}
\end{Highlighting}
\end{Shaded}

\begin{verbatim}
## [1] "The p-value of the test is: 0.282054"
\end{verbatim}

\begin{enumerate}
\def\labelenumi{\arabic{enumi}.}
\setcounter{enumi}{3}
\tightlist
\item
  Make a Decision
\end{enumerate}

p-value is lower than the level of significance (0.10), then we FAIL to
reject \(H_{0}\).

\begin{enumerate}
\def\labelenumi{\arabic{enumi}.}
\setcounter{enumi}{4}
\tightlist
\item
  Interpretation of the results
\end{enumerate}

At the 10\% of significance level there's NO sufficient evidence to
support \(H_{1}\) (our claim)

Extra. Using R for Hypothesis Test

\begin{Shaded}
\begin{Highlighting}[]
\FunctionTok{prop.test}\NormalTok{(}\AttributeTok{x=}\DecValTok{29}\NormalTok{, }\AttributeTok{n=}\DecValTok{321}\NormalTok{, }\AttributeTok{p=}\FloatTok{0.10}\NormalTok{, }\AttributeTok{alternative=}\StringTok{"less"}\NormalTok{)}
\end{Highlighting}
\end{Shaded}

\begin{verbatim}
## 
##  1-sample proportions test with continuity correction
## 
## data:  29 out of 321, null probability 0.1
## X-squared = 0.23399, df = 1, p-value = 0.3143
## alternative hypothesis: true p is less than 0.1
## 95 percent confidence interval:
##  0.0000000 0.1219411
## sample estimates:
##          p 
## 0.09034268
\end{verbatim}

\hypertarget{problem-4}{%
\subsubsection{Problem 4}\label{problem-4}}

Trials in an experiment with a polygraph include 97 results that include
23 cases of wrong results and 74 cases of correct result. Use a 0.05
significance level to test the claim that such polygraph results are
correct less than 80\% of the time. Identify the null hypothesis,
alternative hypothesis, tests statistic, p-value, conclusion about the
null hypothesis and final conclusion that addresses the original claim.
Use the p-value method and the normal distribution as an approximation
of the binomial distribution.

Solution:

\begin{enumerate}
\def\labelenumi{\arabic{enumi}.}
\tightlist
\item
  State null and alternate Hypothesis:
\end{enumerate}

\begin{itemize}
\tightlist
\item
  \(H_{0}\): \(p=0.80\)
\item
  \(H_{1}\): \(p<0.80\) (Left tailed test)
\end{itemize}

\begin{enumerate}
\def\labelenumi{\arabic{enumi}.}
\setcounter{enumi}{1}
\tightlist
\item
  Compute the Test Statistic
\end{enumerate}

\begin{Shaded}
\begin{Highlighting}[]
\NormalTok{p40}\OtherTok{\textless{}{-}}\FloatTok{0.80}
\NormalTok{X4}\OtherTok{\textless{}{-}}\DecValTok{74}
\NormalTok{n4}\OtherTok{\textless{}{-}}\DecValTok{97}
\NormalTok{p4bar}\OtherTok{\textless{}{-}}\NormalTok{X4}\SpecialCharTok{/}\NormalTok{n4}
\NormalTok{z4}\OtherTok{\textless{}{-}}\NormalTok{(p4bar}\SpecialCharTok{{-}}\NormalTok{p40)}\SpecialCharTok{/}\FunctionTok{sqrt}\NormalTok{(p40}\SpecialCharTok{*}\NormalTok{(}\DecValTok{1}\SpecialCharTok{{-}}\NormalTok{p40)}\SpecialCharTok{/}\NormalTok{n4)}
\FunctionTok{sprintf}\NormalTok{(}\StringTok{"The value of the z statistic is: \%f"}\NormalTok{, z4)}
\end{Highlighting}
\end{Shaded}

\begin{verbatim}
## [1] "The value of the z statistic is: -0.913812"
\end{verbatim}

\begin{enumerate}
\def\labelenumi{\arabic{enumi}.}
\setcounter{enumi}{2}
\tightlist
\item
  Compute the p-value
\end{enumerate}

\begin{Shaded}
\begin{Highlighting}[]
\NormalTok{p4\_value}\OtherTok{\textless{}{-}}\FunctionTok{pnorm}\NormalTok{(z4, }\DecValTok{0}\NormalTok{, }\DecValTok{1}\NormalTok{)}
\FunctionTok{sprintf}\NormalTok{(}\StringTok{"The p{-}value of the test is: \%f"}\NormalTok{, p4\_value)}
\end{Highlighting}
\end{Shaded}

\begin{verbatim}
## [1] "The p-value of the test is: 0.180408"
\end{verbatim}

\begin{enumerate}
\def\labelenumi{\arabic{enumi}.}
\setcounter{enumi}{3}
\tightlist
\item
  Make a Decision
\end{enumerate}

p-value is lower than the level of significance (0.05), then we FAIL to
reject \(H_{0}\).

\begin{enumerate}
\def\labelenumi{\arabic{enumi}.}
\setcounter{enumi}{4}
\tightlist
\item
  Interpretation of the results
\end{enumerate}

At the 5\% of significance level there's NO sufficient evidence to
support \(H_{1}\) (our claim)

Extra. Using R for Hypothesis Test

\begin{Shaded}
\begin{Highlighting}[]
\FunctionTok{prop.test}\NormalTok{(}\AttributeTok{x=}\DecValTok{74}\NormalTok{, }\AttributeTok{n=}\DecValTok{97}\NormalTok{, }\AttributeTok{p=}\FloatTok{0.80}\NormalTok{, }\AttributeTok{alternative=}\StringTok{"less"}\NormalTok{)}
\end{Highlighting}
\end{Shaded}

\begin{verbatim}
## 
##  1-sample proportions test with continuity correction
## 
## data:  74 out of 97, null probability 0.8
## X-squared = 0.6192, df = 1, p-value = 0.2157
## alternative hypothesis: true p is less than 0.8
## 95 percent confidence interval:
##  0.0000000 0.8306761
## sample estimates:
##         p 
## 0.7628866
\end{verbatim}

\hypertarget{problem-5}{%
\subsubsection{Problem 5}\label{problem-5}}

Consider a sample of 52 football games, where 31 of them were won by the
home team. Use a 0.10 significance level to test the claim that the
probability that he home team wins is greater than one-half

Solution:

\begin{enumerate}
\def\labelenumi{\arabic{enumi}.}
\tightlist
\item
  State null and alternate Hypothesis:
\end{enumerate}

\begin{itemize}
\tightlist
\item
  \(H_{0}\): \(p=0.50\)
\item
  \(H_{1}\): \(p>0.50\) (Right tailed test)
\end{itemize}

\begin{enumerate}
\def\labelenumi{\arabic{enumi}.}
\setcounter{enumi}{1}
\tightlist
\item
  Compute the Test Statistic
\end{enumerate}

\begin{Shaded}
\begin{Highlighting}[]
\NormalTok{p50}\OtherTok{\textless{}{-}}\FloatTok{0.50}
\NormalTok{X5}\OtherTok{\textless{}{-}}\DecValTok{31}
\NormalTok{n5}\OtherTok{\textless{}{-}}\DecValTok{52}
\NormalTok{p5bar}\OtherTok{\textless{}{-}}\NormalTok{X5}\SpecialCharTok{/}\NormalTok{n5}
\NormalTok{z5}\OtherTok{\textless{}{-}}\NormalTok{(p5bar}\SpecialCharTok{{-}}\NormalTok{p50)}\SpecialCharTok{/}\FunctionTok{sqrt}\NormalTok{(p50}\SpecialCharTok{*}\NormalTok{(}\DecValTok{1}\SpecialCharTok{{-}}\NormalTok{p50)}\SpecialCharTok{/}\NormalTok{n5)}
\FunctionTok{sprintf}\NormalTok{(}\StringTok{"The value of the z statistic is: \%f"}\NormalTok{, z5)}
\end{Highlighting}
\end{Shaded}

\begin{verbatim}
## [1] "The value of the z statistic is: 1.386750"
\end{verbatim}

\begin{enumerate}
\def\labelenumi{\arabic{enumi}.}
\setcounter{enumi}{2}
\tightlist
\item
  Compute the p-value
\end{enumerate}

\begin{Shaded}
\begin{Highlighting}[]
\NormalTok{p5\_value}\OtherTok{\textless{}{-}}\DecValTok{1}\SpecialCharTok{{-}}\FunctionTok{pnorm}\NormalTok{(z5, }\DecValTok{0}\NormalTok{, }\DecValTok{1}\NormalTok{)}
\FunctionTok{sprintf}\NormalTok{(}\StringTok{"The p{-}value of the test is: \%f"}\NormalTok{, p5\_value)}
\end{Highlighting}
\end{Shaded}

\begin{verbatim}
## [1] "The p-value of the test is: 0.082759"
\end{verbatim}

\begin{enumerate}
\def\labelenumi{\arabic{enumi}.}
\setcounter{enumi}{3}
\tightlist
\item
  Make a Decision
\end{enumerate}

p-value is lower than the level of significance (0.1) then we must
reject \(H_{0}\).

\begin{enumerate}
\def\labelenumi{\arabic{enumi}.}
\setcounter{enumi}{4}
\tightlist
\item
  Interpretation of the results
\end{enumerate}

At the 10\% of significance level there are sufficient evidence to
support \(H_{1}\) (our claim)

Extra. Using R for Hypothesis Test

\begin{Shaded}
\begin{Highlighting}[]
\FunctionTok{prop.test}\NormalTok{(}\AttributeTok{x=}\DecValTok{31}\NormalTok{, }\AttributeTok{n=}\DecValTok{52}\NormalTok{, }\AttributeTok{p=}\FloatTok{0.50}\NormalTok{, }\AttributeTok{alternative=}\StringTok{"greater"}\NormalTok{)}
\end{Highlighting}
\end{Shaded}

\begin{verbatim}
## 
##  1-sample proportions test with continuity correction
## 
## data:  31 out of 52, null probability 0.5
## X-squared = 1.5577, df = 1, p-value = 0.106
## alternative hypothesis: true p is greater than 0.5
## 95 percent confidence interval:
##  0.472659 1.000000
## sample estimates:
##         p 
## 0.5961538
\end{verbatim}

In this case the p-value supplied by prop.test function is slightly
greater than the significance level required. This is because it is
using a different test statistic (x-squared). The same happens with the
binom.test function

\begin{Shaded}
\begin{Highlighting}[]
\FunctionTok{binom.test}\NormalTok{(}\AttributeTok{x=}\DecValTok{31}\NormalTok{, }\AttributeTok{n=}\DecValTok{52}\NormalTok{, }\AttributeTok{p=}\FloatTok{0.50}\NormalTok{, }\AttributeTok{alternative=}\StringTok{"greater"}\NormalTok{)}
\end{Highlighting}
\end{Shaded}

\begin{verbatim}
## 
##  Exact binomial test
## 
## data:  31 and 52
## number of successes = 31, number of trials = 52, p-value = 0.1058
## alternative hypothesis: true probability of success is greater than 0.5
## 95 percent confidence interval:
##  0.4726687 1.0000000
## sample estimates:
## probability of success 
##              0.5961538
\end{verbatim}

\end{document}
